\documentclass{article}
\usepackage{cite}
\usepackage[hidelinks]{hyperref}

\newcommand{\pytvversion}{0.5.6}

\title{PyTV: Python Templated Verilog}
\author{Wuqiong Zhao}

\begin{document}

\maketitle

\begin{abstract}
  PyTV (Python Templated Verilog)%
  \footnote{This is the documentation to PyTV version \texttt{\pytvversion{}}.}
  is a Rust library and binary tool that extends Verilog with Python templating.
\end{abstract}

\tableofcontents

\section{Introduction}\label{sec:introduction}
% Templated languages
One method of auto generation of parameterized hardware relies on templated languages.
AHDW \cite{zhao2023automatic} is a language building upon Verilog with custom syntax for parameterized hardware.
% PyTV's advantage

Different from other domain-specific languages (DSLs) for hardware generation,
PyTV provides users with more control over the hardware architecture and implementation details.

% PyTV project information
PyTV is open-source at \url{https://github.com/autohdw/pytv}
and is distributed under the GPL-3.0 license.
The Rust crate is available at \url{https://crates.io/crates/pytv}.


\section{Installation}\label{sec:installation}
\subsection{Installation as a Rust Library}
Similar to other library crates, PyTV can be installed as a Rust library by adding the following line to the \texttt{Cargo.toml} file of your project:
\begin{verbatim}
[dependencies]
pytv = "0.5.6"
\end{verbatim}
The version number can be replaced with the desired version of PyTV.
The library can then be imported into your Rust code by adding the following line:
\begin{verbatim}
extern crate pytv;
\end{verbatim}

You can also install PyTV using the \texttt{cargo} command:
\begin{verbatim}
cargo install pytv
\end{verbatim}

\subsection{Installation as a Binary Tool}
PyTV can also be installed as a binary tool by running the following command:
\begin{verbatim}
cargo install pytv
\end{verbatim}

You can also build the binary tool from source by running the following command:
\begin{verbatim}
cargo build --release
\end{verbatim}

Note that the generation of Verilog files 


\section{Syntax Specifications}\label{sec:syntax}
\subsection{Basics: Python Line and Verilog Line}
As a general rule, a \textit{Python line} is a line of Python code,
and a \textit{Verilog line} is a line of Verilog code.
A minimal example is shown below:

\begin{verbatim}
//! a = 1 + 2;            #  Python inline
assign wire_`a` = wire_b; // Verilog line
/*!
b = a ** 2;               #  Python block
*/
\end{verbatim}

The syntax rule can be summarized as follows:
\begin{itemize}
  \item An \textit{inline Python line} starts with \texttt{//!}.
  \item A \textit{Python block} starts with \texttt{/*!} and ends with \texttt{*/}.
  \item Otherwise, it is a \textit{Verilog line}.
  In the Verilog line, contents in backticks (\verb|`|) are treated as Python variables,
  and are calculated inline.
\end{itemize}

Internally, a Python line will be copied to the generated \texttt{.v.py} file.
A Verilog line will be a \texttt{print} statement,
where contents in backticks (\verb|`|) are properly escaped and embedded in the format string.

\subsection{Indentation}
Because the generation framework is based on Python,
indentation is important.
The mixture of Python and Verilog code adds to the complexity.
Therefore, the following rules are enforced.

\textbf{Rule 1: same with Python.}
The number of spaces for Python indentation is a fixed number as required by Python.
It is recommended to use 4 spaces for Python indentation.

\textbf{Rule 2: no preceding spaces.}
For Python lines, no preceding spaces are allowed before the \texttt{//!}, \texttt{/*!}, or \texttt{*/}.
Otherwise they are not recognized as Python lines or blocks.

\textbf{Rule 3: first line decides for all.}

\textbf{Rule 4: proceeding lines follow.}

\textbf{Rule 5: ease with Verilog.}
The indentation of Verilog lines do not matter.
They will be printed as is.

\textbf{Rule 6: no tabs are allowed.}
Notably, tabs are strongly discouraged in PyTV, as it may lead to unexpected behavior due to indentation mismatch.
Always use spaces for indentation.

\subsection{Instantiation}
One important feature of PyTV is its capability to instantiate modules with complex parameters,
and to enable hierarchy extraction.


\bibliographystyle{IEEEtran}
\phantomsection
\addcontentsline{toc}{section}{References}
\bgroup
\small
\bibliography{IEEEabrv, ref}
\egroup

\end{document}
