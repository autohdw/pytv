\subsection{Basics: Python Line and Verilog Line}
As a general rule, a \textit{Python line} is a line of Python code,
and a \textit{Verilog line} is a line of Verilog code.
A minimal example is shown below:

\begin{verbatim}
//! a = 1 + 2;            #  Python inline
assign wire_`a` = wire_b; // Verilog line
/*!
b = a ** 2;               #  Python block
*/
\end{verbatim}

The syntax rule can be summarized as follows:
\begin{itemize}
  \item An \textit{inline Python line} starts with \texttt{//!}.
  \item A \textit{Python block} starts with \texttt{/*!} and ends with \texttt{*/}.
  \item Otherwise, it is a \textit{Verilog line}.
  In the Verilog line, contents in backticks (\verb|`|) are treated as Python variables,
  and are calculated inline.
\end{itemize}

Internally, a Python line will be copied to the generated \texttt{.v.py} file.
A Verilog line will be a \texttt{print} statement,
where contents in backticks (\verb|`|) are properly escaped and embedded in the format string.

\subsection{Indentation}
Because the generation framework is based on Python,
indentation is important.
The mixture of Python and Verilog code adds to the complexity.
Therefore, the following rules are enforced.

\textbf{Rule 1: same with Python.}
The number of spaces for Python indentation is a fixed number as required by Python.
It is recommended to use 4 spaces for Python indentation.

\textbf{Rule 2: no preceding spaces.}
For Python lines, no preceding spaces are allowed before the \texttt{//!}, \texttt{/*!}, or \texttt{*/}.
Otherwise they are not recognized as Python lines or blocks.

\textbf{Rule 3: first line decides for all.}

\textbf{Rule 4: proceeding lines follow.}

\textbf{Rule 5: ease with Verilog.}
The indentation of Verilog lines do not matter.
They will be printed as is.

\textbf{Rule 6: no tabs are allowed.}
Notably, tabs are strongly discouraged in PyTV, as it may lead to unexpected behavior due to indentation mismatch.
Always use spaces for indentation.

\subsection{Instantiation}
One important feature of PyTV is its capability to instantiate modules with complex parameters,
and to enable hierarchy extraction.
