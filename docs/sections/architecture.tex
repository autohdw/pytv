The architecture (generation process) of PyTV is illustrated in Figure~\ref{fig:architecture}.

\begin{figure}[htbp]
  \centering
  \begin{tikzpicture}[
    , thick
    , font = \sffamily
    , n/.style = {
      , draw
      , fill = gray!10
      , minimum width = 14mm
      , minimum height = 8mm
      , font = \ttfamily
      , text height = 1.3ex
    }
    , node distance = 8mm and 15mm
  ]
    \node (pytv) [n] {.pytv};
    \node (v-py) [n, right = of pytv] {.v.py};
    \node (v) [n, right = of v-py] {.v};
    \node (inst) [n, below = of v-py] {.inst};
    \draw [-latex] (pytv) -- (v-py);
    \draw [-latex] (v-py) -- (v) node [midway, above] {\small python3};
    \draw [-latex] (v-py) -- (inst) node [midway, right] {\small python3};
  \end{tikzpicture}
  \caption{Architecture of PyTV.}
  \label{fig:architecture}
\end{figure}
