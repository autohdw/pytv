\subsection{Installation as a Rust Library}
Similar to other library crates, PyTV can be installed as a Rust library by adding the following line to the \texttt{Cargo.toml} file of your project:
\begin{verbatim}
[dependencies]
pytv = "0.5.6"
\end{verbatim}
The version number can be replaced with the desired version of PyTV.
The library can then be imported into your Rust code by adding the following line:
\begin{verbatim}
extern crate pytv;
\end{verbatim}

You can also install PyTV using the \texttt{cargo} command:
\begin{verbatim}
cargo install pytv
\end{verbatim}

\subsection{Installation as a Binary Tool}
PyTV can also be installed as a binary tool by running the following command:
\begin{verbatim}
cargo install pytv
\end{verbatim}

You can also build the binary tool from source by running the following command:
\begin{verbatim}
cargo build --release
\end{verbatim}

\subsection{Dependencies}
Note that the generation of Verilog and instantiation files requires the use of \textit{Python 3},
so make sure that Python 3 is installed on your system, and accessible in the \texttt{PATH} environment variable.
On Linux and macOS, \texttt{python3} is the command to invoke Python 3,
while on Windows, it is \texttt{python}.
